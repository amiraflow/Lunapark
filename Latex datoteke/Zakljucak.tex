\chapter{Zaključak}

U ovom seminarskom radu prikazano je koji su to koraci potrebni za projektovanje jednog mikroproceskorskog sistema koji funkcioniše u realnom vremenu. Sistem koji je napravljen je lunapark. \\

Lunapark ili park zabave je ograđeni prostor sa nizom elemenata namijenjenih zabavi većeg broja ljudi, te zaradi vlasnika. Pojava određenih ograđenih prostora samo za zabavu postoji još od početka prošlog milenija, a smatra se da su moderni parkovi zabave nastali u 19. stoljeću. Danas su veoma popularni takozvani tematski parkovi, odnosno lunaparkovi koji su napravljeni na neku temu. \\

Za kreiranje ovog sistema, korištena je Arduino Mega pločica, velik broj komponenti, Arduino programsko okruženje, Android Studio programsko okruženje, te Tensorflow Lite neuronska mreža. Ukrasni, nefunkcionalni dio, pravljen je od kartona, te su slike uređivane uz pomoć Adobe Photoshopa. Dobijen je lijep i funkcionalan sistem, ali nesavršen zbog materijala - kartona, te nesavršenih komponenti. \\

Pokretanje sistema realizirano je kao sistem sa ulaznicama. Korisnik treba pokretnuti Android aplikaciju, te preko kamere izvršiti akviziciju slike. Ako je na slici jedna od ulaznica sistema, korisniku se dopušta da se poveže na bluetooth i upravlja pojedinim dijelovima sistema. Prepoznavanje ulaznice istrenirano je pomoću neuronske mreže. Radnje koje korisnik kontrolira su pokretanje i zaustavljanje vrteški, uključivanje i isključivanje svjetala, pokretanje igre, te pokretanje i zaustavljanje muzike. 




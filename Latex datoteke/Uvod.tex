\chapter{Uvod}

U ovom radu prikazan je postupak pri izradi jednog lunaparka kao mikroprocesorskog sistema. Lunapark ili zabavni park je naziv za skup zabavnih sadržaja, vožnji i sličnih zbivanja koji se trajno odvijaju na jednoj lokaciji, namijenjenih za zabavu većeg broja ljudi. Složeniji je od igrališta ili gradskog parka, pošto ima sadržaje za sve starosne skupine.  \\

Za projektovanje lunaparka korištena je pločica Arduino Mega, te veliki broj senzora i motora. Pokretanje sistema realizirano je kao sistem ulaznica sa Android aplikacijom, razvijenom pomoću softvera Android Studio, te Tensorflow Lite, framework otvorenog kôda za duboko učenje koji je namijenjen mobilnim uređajima. Korisnik pokrene aplikaciju, odmah se otvori kamera njegovog mobitela ispred koje treba staviti ulaznicu za sistem. Ulaznice su napravljene posebno za ovaj sistem korištenjem softvera Adobe Photoshop, a prepoznavanje istih urađeno je treniranjem neuronske mreže. Kad aplikacija prepozna ulaznicu, korisniku se automatski pokrene izbornik za povezivanje preko bluetootha. Nakon što se korisnik poveže, otvori mu se ekran s izbornikom za pokretanje i zaustavljanje određenih dijelova sistema, uključivanje i isključivanje svjetala, pokretanje i zaustavljanje vrteški, pokretanje igre, te pokretanje i zaustavljanje muzike. \\

Svi dijelovi lunaparka pravljeni su od kartona u koji su smještene elektronske komponente, te povezane s ostatkom sistema žicama. Glavni dio sistema, veliki šator sa Arduino Mega pločicom, driverima za motore, te bluetooth modulom i modulom za SD karticu, smješten je u sredinu makete. 
\chapter{Programski kôd}

Programski kôd korišten za rad sistema dan je u nastavku.
\begin{lstlisting}[frame=single,language=C++,numbers=left, numberstyle=\tiny, xleftmargin=0.05\textwidth, xrightmargin=0.05\textwidth, basicstyle=\ttfamily\footnotesize]
#include <Servo.h>
#include "SD.h"
#define SD_ChipSelectPin 53
#include "TMRpcm.h"
#include "SPI.h"

// SD kartica
TMRpcm tmrpcm;
// Senzor za boju
#define S0 2
#define S1 3
#define S2 4
#define S3 5
#define sensorOut 6
const int crveniButton = 28;    
const int plaviButton = 26; 
Servo gornjiServo;
Servo donjiServo;

// svjetla
const int crvenaDioda = 29; 
const int zelenaDioda = 30; 

const int dioda1 = 32; 
const int dioda2 = 31; 
const int dioda3 = 33;
const int dioda4 = 34;
const int svjetla = 36;

// aplikacija
long int data;

// motori
#define enA 10
int motor1pin1 = 40;
int motor1pin2 = 41;

int motor2pin1 = 42;
int motor2pin2 = 43;

int motor3pin1 = 44;
int motor3pin2 = 45;

int motor4pin1 = 46;
int motor4pin2 = 47;

// vracara
int crveni_pin= 11;
int zeleni_pin = 12;
int plavi_pin = 13;

// senzor za boju
int frekvencija = 0;
int color = 0;
int crveniButtonState = 0;    
int plaviButtonState = 0; 
// citanje buttona

void setup(){
// pokretanje SD kartice
tmrpcm.speakerPin = 46;
Serial.begin(9600);
if(!SD.begin(SD_ChipSelectPin)){
  Serial.println("SD fail");
  return;
}
tmrpcm.setVolume(6);
    
//bluetooth pokretanje na serijskom portu 1
Serial1.begin(9600);

// senzor za boju
pinMode(crveniButton, INPUT);
pinMode(plaviButton, INPUT);
pinMode(zutiButton, INPUT);
pinMode(S0, OUTPUT);
pinMode(S1, OUTPUT);
pinMode(S2, OUTPUT);
pinMode(S3, OUTPUT);
pinMode(sensorOut, INPUT);

// skaliranje
digitalWrite(S0, HIGH);
digitalWrite(S1, LOW);

gornjiServo.attach(22);
donjiServo.attach(24);

// motori
 pinMode(motor1pin1, OUTPUT);
 pinMode(motor1pin2, OUTPUT);
 pinMode(motor2pin1, OUTPUT);
 pinMode(motor2pin2, OUTPUT);
 pinMode(motor3pin1, OUTPUT);
 pinMode(motor3pin2, OUTPUT);
 pinMode(motor4pin1, OUTPUT);
 pinMode(motor4pin2, OUTPUT);

// vracara
pinMode(crveni_pin, OUTPUT);
pinMode(zeleni_pin, OUTPUT);
pinMode(plavi_pin, OUTPUT);
 
// DIODE
pinMode(zelenaDioda, OUTPUT);
pinMode(crvenaDioda, OUTPUT);
pinMode(dioda1, OUTPUT);
pinMode(dioda2, OUTPUT);
pinMode(dioda3, OUTPUT);
pinMode(dioda4, OUTPUT);
pinMode(svjetla, OUTPUT);
    
void loop(){
 
while(Serial1.available()==0) ;

donjiServo.write(0);
if(Serial1.available()>0) {
data = Serial1.parseInt();
 
} 

delay(400);
Serial.print(data);
 
if (data == 67){
  diodePali();
}
if (data == 76){
  diodeGasi();
}
if (data == 23){
  vracara();
}
if (data == 32){
  vracaraUgasi();
}
if (data == 56){
  analogWrite(enA, 160); 
  digitalWrite(motor4pin1, HIGH);
  digitalWrite(motor4pin2, LOW);
}
if (data == 65){
  digitalWrite(motor4pin1, LOW);
  digitalWrite(motor4pin2, LOW);
}
if (data == 34){
  digitalWrite(motor1pin1, LOW);
  digitalWrite(motor1pin2, HIGH);
}
if (data == 43){
  digitalWrite(motor1pin1, LOW);
  digitalWrite(motor1pin2, LOW);
}
if (data == 11){ 
  digitalWrite(motor3pin1, HIGH);
  digitalWrite(motor3pin2, LOW);
  digitalWrite(motor2pin1, LOW);
  digitalWrite(motor2pin2, HIGH);
}
if (data == 21){
  digitalWrite(motor3pin1, LOW);
  digitalWrite(motor3pin2, LOW);
  digitalWrite(motor2pin1, LOW);
  digitalWrite(motor2pin2, LOW);
}
if (data == 91){ 
   tmrpcm.play("test.wav");
}
if (data == 19){
   tmrpcm.pause();
}
if (data == 89){ 
   crveniButtonState = digitalRead(crveniButton);
   plaviButtonState = digitalRead(plaviButton);

if (crveniButtonState == HIGH || zutiButtonState == HIGH || plaviButtonState == HIGH) {
    senzorZaBoje(crveniButtonState, zutiButtonState, plaviButtonState);
}
}
if (data == 98){
   donjiServo.write(110);
}
if (data == 99){ 
   crveniButtonState = digitalRead(crveniButton);
   plaviButtonState = digitalRead(plaviButton);

if (crveniButtonState == HIGH || zutiButtonState == HIGH || plaviButtonState == HIGH) {
    senzorZaBoje(crveniButtonState, zutiButtonState, plaviButtonState);
}
 diodePali();
   digitalWrite(motor1pin1, LOW);
   digitalWrite(motor1pin2, HIGH);
  digitalWrite(motor3pin1, HIGH);
  digitalWrite(motor3pin2, LOW);
  digitalWrite(motor2pin1, LOW);
  digitalWrite(motor2pin2, HIGH);
  analogWrite(enA, 160); 
  digitalWrite(motor4pin1, HIGH);
  digitalWrite(motor4pin2, LOW);
  vracara();
  tmrpcm.play("test.wav");
}
if (data == 88){
  diodeGasi();
  donjiServo.write(110);
  digitalWrite(motor1pin1, LOW);
  digitalWrite(motor1pin2, LOW);
  digitalWrite(motor3pin1, LOW);
  digitalWrite(motor3pin2, LOW);
  digitalWrite(motor2pin1, LOW);
  digitalWrite(motor2pin2, LOW);
  analogWrite(enA, 160); 
  digitalWrite(motor4pin1, LOW);
  digitalWrite(motor4pin2, LOW);
  vracaraUgasi();
  tmrpcm.pause();
}

}
void vracara(){
  RGB_color(255, 0, 0); // crvena
  delay(1000);
  RGB_color(0, 255, 0); // zelena
  delay(1000);
  RGB_color(0, 0, 255); // plava
  delay(1000);
  RGB_color(255, 255, 125); // roza
  delay(1000);
  RGB_color(0, 255, 255); // cijan
  delay(1000);
  RGB_color(255, 0, 255); // magenta
  delay(1000);
  RGB_color(255, 255, 0); // zuta
  delay(1000);
  RGB_color(255, 255, 255); // bijela
  delay(1000);
}


void RGB_color(int crvena, int zelena, int plava) {
  analogWrite(crveni_pin, crvena);
  analogWrite(zeleni_pin, zelena);
  analogWrite(plavi_pin, plava);
}

int readColor() {
// Citanje crvenih fotodioda
  digitalWrite(S2, LOW);
  digitalWrite(S3, LOW);
  frekvencija = pulseIn(sensorOut, LOW);
  int R = frekvencija;
    delay(50);

// Citanje zelenih fotodioda
  digitalWrite(S2, HIGH);
  digitalWrite(S3, HIGH);
  frekvencija = pulseIn(sensorOut, LOW);
  int G = frekvencija;
  delay(50);
  
// Citanje plavih
  digitalWrite(S2, LOW);
  digitalWrite(S3, HIGH);
  frekvencija = pulseIn(sensorOut, LOW);
  int B = frekvencja;
  delay(50);

  if((R<45 & R>32 & G<70 & G>50) || (R>70 & G<10 & B <50 & B>30) ){
    color = 1; // Crvena
  }
  if(R<45 & R>35 & B<40 &B>30){
    color = 2; // Roza
  }
  if(R<63 & R>50 & G<55 & G>45 & B<40 & B>29){
    color = 3; // Plava
  }
  if(R<40 & R>30 & G<50 & G>40 & B<46 & B>38){
    color = 4; // Zuta
  }
  return color;  
}

void senzorZaBoje(){
   crveniButtonState = digitalRead(crveniButton);
   plaviButtonState = digitalRead(plaviButton);

// Ako je pritisnut jedan od buttona, pokrese se dio programa. Ako korisnik pritisne odgovarajuci button, te pogodi koje je boje loptica, on pobijedi, pali se zelena dioda, u suprotnom, pali se crvena.
  if (crveniButtonState == HIGH || plaviButtonState == HIGH) {
  gornjiServo.write(120);
  delay(3000);
  
  for(int i = 115; i > 55; i--) {
    gornjiServo.write(i);
    delay(2);
  }
  delay(1000);
  
  color = readColor();
  color = readColor();
  color = readColor();
  color = readColor();
  
  delay(1000);  
   switch (color) {
    case 1: //crvena
    if (crveniButtonState == HIGH){
      digitalWrite(zelenaDioda, HIGH);
      digitalWrite(crvenaDioda, LOW);
    }
    if (plaviButtonState == HIGH ){
      digitalWrite(crvenaDioda, HIGH);
      digitalWrite(zelenaDioda, LOW);
    }
    donjiServo.write(50);
    break;
    case 2: //roza
    digitalWrite(crvenaDioda, HIGH);
    digitalWrite(zelenaDioda, LOW);
    donjiServo.write(75);
    break;
    case 3: // zuta
      digitalWrite(crvenaDioda, HIGH);
      digitalWrite(zelenaDioda, LOW);
    donjiServo.write(100);
    break;
    case 4: // plava
     if (plaviButtonState == HIGH){
    digitalWrite(zelenaDioda, HIGH);
    digitalWrite(crvenaDioda, LOW);
    }
    if (crveniButtonState == HIGH ){
    digitalWrite(crvenaDioda, HIGH);
    digitalWrite(zelenaDioda, LOW);
    }
    donjiServo.write(125);
    break;  
    case 0:
    break;
  }
  delay(300);
  
  for(int i = 55; i > 20; i--) {
    gornjiServo.write(i);
    delay(2);
  } 
  delay(500);
  
  for(int i = 29; i < 115; i++) {
    gornjiServo.write(i);
    delay(2);
  }
  color=0;
  }
}

void vrtiMotore(){
  digitalWrite(motor1pin1, HIGH);
  digitalWrite(motor1pin2, LOW);

  digitalWrite(motor2pin1, HIGH);
  digitalWrite(motor2pin2, LOW);

  digitalWrite(motor3pin1, HIGH);
  digitalWrite(motor3pin2, LOW);
  
    analogWrite(enA, 160);  // upravljanje brzinom
  digitalWrite(motor4pin1, HIGH);
  digitalWrite(motor4pin2, LOW);
}
 \end{lstlisting}